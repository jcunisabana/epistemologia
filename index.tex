\documentclass[spanish,]{article}
\usepackage{lmodern}
\usepackage{amssymb,amsmath}
\usepackage{ifxetex,ifluatex}
\usepackage{fixltx2e} % provides \textsubscript
\ifnum 0\ifxetex 1\fi\ifluatex 1\fi=0 % if pdftex
  \usepackage[T1]{fontenc}
  \usepackage[utf8]{inputenc}
\else % if luatex or xelatex
  \ifxetex
    \usepackage{mathspec}
  \else
    \usepackage{fontspec}
  \fi
  \defaultfontfeatures{Ligatures=TeX,Scale=MatchLowercase}
\fi
% use upquote if available, for straight quotes in verbatim environments
\IfFileExists{upquote.sty}{\usepackage{upquote}}{}
% use microtype if available
\IfFileExists{microtype.sty}{%
\usepackage[]{microtype}
\UseMicrotypeSet[protrusion]{basicmath} % disable protrusion for tt fonts
}{}
\PassOptionsToPackage{hyphens}{url} % url is loaded by hyperref
\usepackage[unicode=true]{hyperref}
\hypersetup{
            pdftitle={Epistemología de la investigación},
            pdfauthor={Una introducción a la epistemología de la interdisciplinariedad},
            pdfborder={0 0 0},
            breaklinks=true}
\urlstyle{same}  % don't use monospace font for urls
\usepackage[margin=1in]{geometry}
\ifnum 0\ifxetex 1\fi\ifluatex 1\fi=0 % if pdftex
  \usepackage[shorthands=off,main=spanish]{babel}
\else
  \usepackage{polyglossia}
  \setmainlanguage[]{spanish}
\fi
\usepackage{graphicx,grffile}
\makeatletter
\def\maxwidth{\ifdim\Gin@nat@width>\linewidth\linewidth\else\Gin@nat@width\fi}
\def\maxheight{\ifdim\Gin@nat@height>\textheight\textheight\else\Gin@nat@height\fi}
\makeatother
% Scale images if necessary, so that they will not overflow the page
% margins by default, and it is still possible to overwrite the defaults
% using explicit options in \includegraphics[width, height, ...]{}
\setkeys{Gin}{width=\maxwidth,height=\maxheight,keepaspectratio}
\IfFileExists{parskip.sty}{%
\usepackage{parskip}
}{% else
\setlength{\parindent}{0pt}
\setlength{\parskip}{6pt plus 2pt minus 1pt}
}
\setlength{\emergencystretch}{3em}  % prevent overfull lines
\providecommand{\tightlist}{%
  \setlength{\itemsep}{0pt}\setlength{\parskip}{0pt}}
\setcounter{secnumdepth}{0}
% Redefines (sub)paragraphs to behave more like sections
\ifx\paragraph\undefined\else
\let\oldparagraph\paragraph
\renewcommand{\paragraph}[1]{\oldparagraph{#1}\mbox{}}
\fi
\ifx\subparagraph\undefined\else
\let\oldsubparagraph\subparagraph
\renewcommand{\subparagraph}[1]{\oldsubparagraph{#1}\mbox{}}
\fi

% set default figure placement to htbp
\makeatletter
\def\fps@figure{htbp}
\makeatother

\usepackage{etoolbox}
\makeatletter
\providecommand{\subtitle}[1]{% add subtitle to \maketitle
  \apptocmd{\@title}{\par {\large #1 \par}}{}{}
}
\makeatother
\usepackage{fontspec}
\setmainfont{Adobe Jenson Pro}
\linespread{1.05}
% https://github.com/rstudio/rmarkdown/issues/337
\let\rmarkdownfootnote\footnote%
\def\footnote{\protect\rmarkdownfootnote}

% https://github.com/rstudio/rmarkdown/pull/252
\usepackage{titling}
\setlength{\droptitle}{-2em}

\pretitle{\vspace{\droptitle}\centering\huge}
\posttitle{\par}

\preauthor{\centering\large\emph}
\postauthor{\par}

\predate{\centering\large\emph}
\postdate{\par}
\usepackage{booktabs}
\usepackage{longtable}
\usepackage{array}
\usepackage{multirow}
\usepackage{wrapfig}
\usepackage{float}
\usepackage{colortbl}
\usepackage{pdflscape}
\usepackage{tabu}
\usepackage{threeparttable}
\usepackage{threeparttablex}
\usepackage[normalem]{ulem}
\usepackage{makecell}
\usepackage{xcolor}

\title{Epistemología de la investigación}
\author{Una introducción a la epistemología de la interdisciplinariedad}
\date{}

\begin{document}
\maketitle

\subsection{Descripción del
seminario}\label{descripciuxf3n-del-seminario}

El objetivo general de este seminario es examinar los métodos,
metodologías y competencias para investigar en el campo que comprende la
unión de las ciencias experimentales, sociales y humanas usualmente
conocido como \emph{interdisciplinariedad} con miras a la integración de
consideraciones sobre cultura, fe y ciencia. Estudiaremos la idea de una
disciplina, su forma de investigar y cómo se puede caracterizar la
investigación que no se ajusta a tales maneras. En el contexto de la
\href{https://www.unisabana.edu.co/index.php?id=12310}{Maestría en
Teología}, la idea es ofrecer al estudiante las bases para una
investigación que no solo abarque las diversas áreas de la Teología,
sino también de las ciencias sociales, humanas o experimentales.

Discutiremos los valores de una disciplina, la forma en que las
metodologías propias de una disciplina deben relacionarse con otras, la
noción de evidencia, la relación entre la investigación, el sujeto y el
objeto de estudio, la necesidad de la neutralidad en la investigación y
la relación de la investigación académica con los llamados ``problemas
reales''.

Para ello, primero examinaremos el rol general de la discusión
epistemológica en la maestría y hablaremos sobre las disciplinas base de
quienes conmpongan el seminario, estas son aquellas que estudiaron
durante su pregrado o la mayor parte de sus estudios, en la que se han
desempeñado principalmente o con la que mejor se identifican. Después
utilizaremos un cuestionario para guiar la discusión sobre la forma en
que se investiga en cada una de sus disciplinas (basado en Hubbs, G.,
O'Rourke, M., \& Orzack, S. H. (2020)) y la manera en que creen que esto
enmarca también su futura investigación a lo largo de la Maestría en
Teología. Posteriormente, discutiremos la tensión en la investigación
disciplinar y su aplicabilidad a los ``problemas reales'' (con un
espacio en donde discutiremos también la idoneidad de la noción de
``problemas reales''). Para finalizar hablaremos sobre la idea misma de
la disciplinariedad y las formas en que podemos caracterizar la
diferencias con la interdisciplinariedad, transdisciplinariedad y
multidisciplinariedad y otras formas de investigación por fuera de los
esquemas presentados por una disciplina. Terminaremos con un retorno al
cuestionario inicial con miras a determinar la forma en que su
aproximación a la investigación ha cambiado a luz del seminario y la
forma en que formularía una investigación sobre fe, ciencia y cultura
dentro del marco epistemológico presentado.

\textbf{Profesor}: \href{../index.html}{Juan Camilo Espejo-Serna}~

\textbf{Página web}: \url{http://jcunisabana.github.io/epistemologia}

\textbf{Horario}: Lunes 18 - Sábado 23, Enero 2021, 4:00 - 6:00 pm

\textbf{Grupo de MS Teams}: \url{https://tinyurl.com/yyf7mjr9}

\subsection{Objetivos}\label{objetivos}

\begin{itemize}
\item
  Leer críticamente textos académicos que permitan desarrollar hábitos
  intelectuales.
\item
  Discernir el ámbito general del tipo de investigación que realizará a
  lo largo del programa académico.
\item
  Producir un ensayo corto (máximo 2000 palabras) en donde formule una
  pregunta bien definida y determine la metodología que va a seguir para
  responderla.
\end{itemize}

\subsection{Sesiones}\label{sesiones}

\subsubsection{Sesión 1}\label{sesiuxf3n-1}

\begin{tabular}{>{\raggedright\arraybackslash}p{30em}|>{\raggedright\arraybackslash}p{30em}}
\hline
Tema & Pregunta central\\
\hline
Presentaci<U+00F3>n general del curso y del cuestionario sobre su disciplina & <U+00BF>Qu<U+00E9> buscan con sus estudios de maestr<U+00ED>a?\\
\hline
\end{tabular}

Presentación en pantalla completa

\begin{center}\rule{0.5\linewidth}{\linethickness}\end{center}

\subsubsection{Semana 2}\label{semana-2}

\begin{tabular}{>{\raggedright\arraybackslash}p{30em}|>{\raggedright\arraybackslash}p{30em}}
\hline
Tema & Pregunta central\\
\hline
Discusi<U+00F3>n en clase con base Michael O<U+2019>Rourke \& Stephen Crowley (2020) y Marisa Rinkus \& Stephanie Vasko (2020) & <U+00BF>C<U+00F3>mo es su disciplina base?\\
\hline
\end{tabular}

\begin{center}\rule{0.5\linewidth}{\linethickness}\end{center}

\subsubsection{Semana 3}\label{semana-3}

\begin{tabular}{>{\raggedright\arraybackslash}p{30em}|>{\raggedright\arraybackslash}p{30em}}
\hline
Tema & Pregunta central\\
\hline
Presentaci<U+00F3>n y discusi<U+00F3>n con base en Wolfagan Krohn (2017) & <U+00BF>Qu<U+00E9> retos enfrenta la investigaci<U+00F3>n interdisciplinar?\\
\hline
\end{tabular}

\begin{center}\rule{0.5\linewidth}{\linethickness}\end{center}

\subsubsection{Semana 4}\label{semana-4}

\begin{tabular}{>{\raggedright\arraybackslash}p{30em}|>{\raggedright\arraybackslash}p{30em}}
\hline
Tema & Pregunta central\\
\hline
La disciplinariedad con base en Richard Frodeman (2013) Cap 2 & <U+00BF>C<U+00F3>mo se investiga en su disciplina base?\\
\hline
\end{tabular}

\begin{center}\rule{0.5\linewidth}{\linethickness}\end{center}

\subsubsection{Semana 5}\label{semana-5}

\begin{tabular}{>{\raggedright\arraybackslash}p{30em}|>{\raggedright\arraybackslash}p{30em}}
\hline
Tema & Pregunta central\\
\hline
La interdisciplinariedad con base en Richard Frodeman (2013) Cap 3 & <U+00BF>Qu<U+00E9> cambios implica la investigaci<U+00F3>n interdisciplinar?\\
\hline
\end{tabular}

\begin{center}\rule{0.5\linewidth}{\linethickness}\end{center}

\subsubsection{Semana 6}\label{semana-6}

\begin{tabular}{>{\raggedright\arraybackslash}p{30em}|>{\raggedright\arraybackslash}p{30em}}
\hline
Tema & Pregunta central\\
\hline
Revisi<U+00F3>n de la discusi<U+00F3>n con base Michael O<U+2019>Rourke \& Stephen Crowley (2020) y Marisa Rinkus \& Stephanie Vasko (2020) & <U+00BF>C<U+00F3>mo formular una pregunta de investigaci<U+00F3>n?\\
\hline
\end{tabular}

\subsection{Evaluación}\label{evaluaciuxf3n}

\paragraph{\texorpdfstring{\textbf{Reseña
crítica}}{Reseña crítica}}\label{reseuxf1a-cruxedtica}

Después de la segunda semana de \emph{Introducción a la Sagrada
Escritura} deberán tomar uno de los textos de la bibliografía sobre
interdisciplinariedad y relacionar ese tipo de consideraciones con el
estudio de una cuestion de fe, ciencia y cultura.

\paragraph{\texorpdfstring{\textbf{Participación
activa}}{Participación activa}}\label{participaciuxf3n-activa}

El diálogo es esencial para el desarrollo de la materia pues será
gracias a éste que podremos ver en acción los retos que presenta la
integración de diferentes perspectivas.

Habrá tres actividades en las que tendrán que participar, ofrenciendo su
punto de vista de manera argumentada y dialogando con las opiniones de
sus compañeros.

\paragraph{\texorpdfstring{\textbf{Ensayo}}{Ensayo}}\label{ensayo}

Extensión: entre 1000 y 2000 palabras.

Se deberán tomar los temas de disciplinariedad e interdisciplininariedad
y adptarlos a un ensayo corto en donde se presente una pregunta
teológica bien definida y una sólida y precisa caracterización de la
metología qu emplearían para responderla.

\paragraph{\texorpdfstring{\textbf{Incumplimiento}}{Incumplimiento}}\label{incumplimiento}

\emph{La vida nos da sorpresas; sorpresas nos da la vida.} Por eso, si
por alguna razón no pueden cumplir con las fechas exigidas para los
trabajos, es importante avisar al profesor con tiempo. Hablemos. No me
tienen que contar todos los detalles de sus problemas pero es importante
que si se encuentran en una situación en la que ven que no pueden
cumplir con los requerimientos del seminario me avisen con la mayor
anticipación posible y encontremos un plan para solventar el problema en
lo que respecta a la clase. Insisto: hablemos, no se pierdan \textbf{:)}
.

\paragraph{\texorpdfstring{\textbf{Calificación}}{Calificación}}\label{calificaciuxf3n}

\begin{tabular}{l|l|l}
\hline
Rese<U+00F1>a cr<U+00ED>tica & Participaci<U+00F3>n activa & Ensayo\\
\hline
15\% & 15\% & 70\%\\
\hline
\end{tabular}

\end{document}
